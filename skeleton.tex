
\documentclass[12pt,halfline,a4paper,]{ouparticle}

% Packages I think are necessary for basic Rmarkdown functionality
\usepackage{hyperref}
\usepackage{graphicx}
\usepackage{listings}
\usepackage{xcolor}
\usepackage{fancyvrb}
\usepackage{framed}

% Link coloring
\hypersetup{breaklinks=true,
            bookmarks=true,
            pdfauthor={},
            pdftitle={STA2005S - Experimental Design Assignment}
            }


%% To allow better options for figure placement
%\usepackage{float}

% Packages that are supposedly required by OUP sty file
\usepackage{amssymb, amsmath, geometry, amsfonts, verbatim, endnotes, setspace}

% use upquote if available, for straight quotes in verbatim environments
\IfFileExists{upquote.sty}{\usepackage{upquote}}{}

% Macros for dealing with affiliations, footnotes, etc.
\makeatletter
\def\Newlabel#1#2#3{\expandafter\gdef\csname #1@#2\endcsname{#3}}

\def\Ref#1#2{\@ifundefined{#1@#2}{???}{\csname #1@#2\endcsname}}

\newcommand*\samethanks[1][\value{footnote}]{\footnotemark[#1]}

\newcommand*\ifcounter[1]{%
  \ifcsname c@#1\endcsname
    \expandafter\@firstoftwo
  \else
    \expandafter\@secondoftwo
  \fi
}

\newcommand*\thanksbycode[1]{%
  \ifcounter{FNCT@#1}
    {\samethanks[\value{FNCT@#1}]}
    {\thanks{\Ref{FN}{#1}}\newcounter{FNCT@#1}\setcounter{FNCT@#1}{\value{footnote}}}
}

% Create labels for Addresses if the are given in Elsevier format

% Create labels for Footnotes if the are given in Elsevier format

% Part for setting citation format package: natbib

% Part for setting citation format package: biblatex

% Pandoc syntax highlighting
\usepackage{color}
\usepackage{fancyvrb}
\newcommand{\VerbBar}{|}
\newcommand{\VERB}{\Verb[commandchars=\\\{\}]}
\DefineVerbatimEnvironment{Highlighting}{Verbatim}{commandchars=\\\{\}}
% Add ',fontsize=\small' for more characters per line
\usepackage{framed}
\definecolor{shadecolor}{RGB}{248,248,248}
\newenvironment{Shaded}{\begin{snugshade}}{\end{snugshade}}
\newcommand{\AlertTok}[1]{\textcolor[rgb]{0.94,0.16,0.16}{#1}}
\newcommand{\AnnotationTok}[1]{\textcolor[rgb]{0.56,0.35,0.01}{\textbf{\textit{#1}}}}
\newcommand{\AttributeTok}[1]{\textcolor[rgb]{0.13,0.29,0.53}{#1}}
\newcommand{\BaseNTok}[1]{\textcolor[rgb]{0.00,0.00,0.81}{#1}}
\newcommand{\BuiltInTok}[1]{#1}
\newcommand{\CharTok}[1]{\textcolor[rgb]{0.31,0.60,0.02}{#1}}
\newcommand{\CommentTok}[1]{\textcolor[rgb]{0.56,0.35,0.01}{\textit{#1}}}
\newcommand{\CommentVarTok}[1]{\textcolor[rgb]{0.56,0.35,0.01}{\textbf{\textit{#1}}}}
\newcommand{\ConstantTok}[1]{\textcolor[rgb]{0.56,0.35,0.01}{#1}}
\newcommand{\ControlFlowTok}[1]{\textcolor[rgb]{0.13,0.29,0.53}{\textbf{#1}}}
\newcommand{\DataTypeTok}[1]{\textcolor[rgb]{0.13,0.29,0.53}{#1}}
\newcommand{\DecValTok}[1]{\textcolor[rgb]{0.00,0.00,0.81}{#1}}
\newcommand{\DocumentationTok}[1]{\textcolor[rgb]{0.56,0.35,0.01}{\textbf{\textit{#1}}}}
\newcommand{\ErrorTok}[1]{\textcolor[rgb]{0.64,0.00,0.00}{\textbf{#1}}}
\newcommand{\ExtensionTok}[1]{#1}
\newcommand{\FloatTok}[1]{\textcolor[rgb]{0.00,0.00,0.81}{#1}}
\newcommand{\FunctionTok}[1]{\textcolor[rgb]{0.13,0.29,0.53}{\textbf{#1}}}
\newcommand{\ImportTok}[1]{#1}
\newcommand{\InformationTok}[1]{\textcolor[rgb]{0.56,0.35,0.01}{\textbf{\textit{#1}}}}
\newcommand{\KeywordTok}[1]{\textcolor[rgb]{0.13,0.29,0.53}{\textbf{#1}}}
\newcommand{\NormalTok}[1]{#1}
\newcommand{\OperatorTok}[1]{\textcolor[rgb]{0.81,0.36,0.00}{\textbf{#1}}}
\newcommand{\OtherTok}[1]{\textcolor[rgb]{0.56,0.35,0.01}{#1}}
\newcommand{\PreprocessorTok}[1]{\textcolor[rgb]{0.56,0.35,0.01}{\textit{#1}}}
\newcommand{\RegionMarkerTok}[1]{#1}
\newcommand{\SpecialCharTok}[1]{\textcolor[rgb]{0.81,0.36,0.00}{\textbf{#1}}}
\newcommand{\SpecialStringTok}[1]{\textcolor[rgb]{0.31,0.60,0.02}{#1}}
\newcommand{\StringTok}[1]{\textcolor[rgb]{0.31,0.60,0.02}{#1}}
\newcommand{\VariableTok}[1]{\textcolor[rgb]{0.00,0.00,0.00}{#1}}
\newcommand{\VerbatimStringTok}[1]{\textcolor[rgb]{0.31,0.60,0.02}{#1}}
\newcommand{\WarningTok}[1]{\textcolor[rgb]{0.56,0.35,0.01}{\textbf{\textit{#1}}}}

% tightlist command for lists without linebreak
\providecommand{\tightlist}{%
  \setlength{\itemsep}{0pt}\setlength{\parskip}{0pt}}



\usepackage{booktabs}

\begin{document}

\title{STA2005S - Experimental Design Assignment}

\author{%
%
% Code for old style authors field
%
% Add \and if both authors and author
%
%
% Code for new (elsevier) style author field
\name{Jing Yeh}
\address{\Ref{ADR}{University of Cape Town}}
%
\email{\href{mailto:yhxjin001@myuct.ac.za}{yhxjin001@myuct.ac.za}}%
%
%
%
\and
\name{Saurav Sathnarayan}
\address{\Ref{ADR}{University of Cape Town}}
%
\email{\href{mailto:yhxjin001@myuct.ac.za}{yhxjin001@myuct.ac.za}}%
%
%
%
%
}

\abstract{This is the abstract.

It consists of two paragraphs.}

\date{2024-09-12}

\keywords{key; dictionary; word}

\maketitle



\newpage

\section{Introduction}\label{introduction}

The goal of this experiment is to identify the programming language that
delivers the fastest execution time when calculating a value of \(\pi\)
with respect to Leibiniz formula. \[\sum_{n=0}^{\infty} (-1)^n/(2n+1)\]

With the increasing demand for high-performance applications,
understanding which programming languages offer superior speed in terms
of execution is crucial for developers, especially in domains requiring
real-time processing, large-scale data analysis, and resource-intensive
computations.\\
This problem will focus on evaluating a selection of popular programming
languages, including but not limited to C++, C, R, Python, Java, and
Ruby. The evaluation will consider how quickly a value of pi can be
calculated by applying leibiniz formula up to 100000000 terms.\\

\paragraph{Compiled Language:}\label{compiled-language}

\hfill\break
In a compiled language, the source code is translated into machine code
by a compiler before execution. This machine code, often called an
executable, can be run directly by the computer's hardware.\\
Compiled programs typically run faster since they are already in machine
language, which the computer's processor can execute directly.\\
Examples: C, C++, Rust, and Go are examples of compiled languages.

\paragraph{Interpreted Language:}\label{interpreted-language}

\hfill\break
In an interpreted language, the source code is executed line-by-line by
an interpreter at runtime. The interpreter reads the code, translates it
into machine code, and executes it on the fly.\\
Interpreted programs generally run slower than compiled ones because the
translation happens during execution.\\
Examples: Python, JavaScript, Ruby, and PHP are examples of interpreted
languages.

\paragraph{Key Differences:}\label{key-differences}

Compiled languages require a compilation step that produces an
executable, while interpreted languages are executed directly by an
interpreter.\\
Compiled languages tend to have better performance due to the
pre-compiled nature of the code, whereas interpreted languages are more
flexible but slower due to the runtime translation.\\
Some languages, like Java, use a combination of both techniques, where
the code is first compiled into an intermediate form (bytecode) and then
interpreted just-in-time (JIT) at runtime.\\
\emph{still need to edit this}

\newpage

\section{Reference example}\label{reference-example}

Here are two sample references:. Bibliography will appear at the end of
the document.

\section{Materials and methods}\label{materials-and-methods}

An equation with a label for cross-referencing:

\begin{equation}\label{eq:eq1}
\sum_{n=0}^{\infty} (-1)^n/(2n+1)
\end{equation}

\subsection{A subsection}\label{a-subsection}

A numbered list:

\begin{enumerate}
\def\labelenumi{\arabic{enumi})}
\tightlist
\item
  First point
\item
  Second point

  \begin{itemize}
  \tightlist
  \item
    Subpoint
  \end{itemize}
\end{enumerate}

A bullet list:

\begin{itemize}
\tightlist
\item
  First point
\item
  Second point
\end{itemize}

\section{Results}\label{results}

computers of different specifications are harder to come by, we will
only use 3 different hardware setup for this pilot study.

\subsection{Pilot Study}\label{pilot-study}

\subsubsection{Problem: Runtimes of Programming Langugaes are not
Normally
Distributed}\label{problem-runtimes-of-programming-langugaes-are-not-normally-distributed}

Upon inspecting the qqnorm plots of the runtime of all programming
languages on the same machine, we discovered that the runtimes of
languages clearly do not follow a normal distribution, even when the
sample size is fairly large (n = 100)

To address this, we ran the program 15 times per sample for each
programming language, and repeated the process 30 times. By the Central
Limit Theorem (CLT), the distribution of sample means is approximately
normal {[}2{]}. If we assume sample means to be normally distributed,
the mean of the distribution of sample means is then an unbiased
estimator for the true run time of each programming language{[}2{]},
which we take as a single observation.

The process described above is automated using the following command:

\begin{verbatim}
python3 run.py main 100000000 15 30
\end{verbatim}

\includegraphics[width=1\linewidth,height=0.45\textheight]{skeleton_files/figure-latex/unnamed-chunk-1-1}

\subsubsection{Problem: Ensuring no Interactions between Blocking
Factors and Treatment
Factors}\label{problem-ensuring-no-interactions-between-blocking-factors-and-treatment-factors}

\emph{todo: insert table for results}

\begin{verbatim}
## Warning in read.table(file = file, header = header, sep = sep, quote = quote, :
## incomplete final line found by readTableHeader on 'pilotData.csv'
\end{verbatim}

\includegraphics[width=1\linewidth]{skeleton_files/figure-latex/unnamed-chunk-2-1}
From the data collected, we observed that the results collected from
Ishango do not follow the general trends established by the other three
setups. Firstly, the hardware setup in Ishango lab is significantly less
advance than MiddleTRNew. Yet, most programming languages tend to
perform better on the Ishango machine. Secondly, to add to the first
observation, not all programming languages perform better on the Ishango
machine.

After further investigation, we learned that programming languages
perform differently on various operating systems {[}4{]}. We
hypothesised that this is likely the reason for the deviation, though
further studies are needed to confirm this (we lack access to machines
with the same hardware setup but run on different operating system).

Therefore, we added another constraint for selecting suitable machines:
the machines must all run on Windows 10, as these machines are the most
widely available. \#\# Generate a figure.

\begin{Shaded}
\begin{Highlighting}[]
\FunctionTok{plot}\NormalTok{(}\DecValTok{1}\SpecialCharTok{:}\DecValTok{10}\NormalTok{, }\AttributeTok{main =} \StringTok{"Some data"}\NormalTok{, }\AttributeTok{xlab =} \StringTok{"Distance (cm)"}\NormalTok{, }\AttributeTok{ylab =} \StringTok{"Time (hours)"}\NormalTok{)}
\end{Highlighting}
\end{Shaded}

\begin{figure}[p]
\includegraphics[width=1\linewidth]{skeleton_files/figure-latex/fig1-1} \caption{This is the first figure.}\label{fig:fig1}
\end{figure}

You can reference this figure as follows: Fig. \ref{fig:fig1}.

\begin{Shaded}
\begin{Highlighting}[]
\FunctionTok{plot}\NormalTok{(}\DecValTok{1}\SpecialCharTok{:}\DecValTok{5}\NormalTok{, }\AttributeTok{pch =} \DecValTok{19}\NormalTok{, }\AttributeTok{main =} \StringTok{"Some data"}\NormalTok{, }\AttributeTok{xlab =} \StringTok{"Distance (cm)"}\NormalTok{, }\AttributeTok{ylab =} \StringTok{"Time (hours)"}\NormalTok{)}
\end{Highlighting}
\end{Shaded}

\begin{figure}[p]
\includegraphics[width=1\linewidth]{skeleton_files/figure-latex/fig2-1} \caption{This is the second figure.}\label{fig:fig2}
\end{figure}

Reference to second figure: Fig. \ref{fig:fig2}

\subsection{\texorpdfstring{Generate a table using
\texttt{xtable}}{Generate a table using xtable}}\label{generate-a-table-using-xtable}

\begin{Shaded}
\begin{Highlighting}[]
\NormalTok{df }\OtherTok{\textless{}{-}} \FunctionTok{data.frame}\NormalTok{(}\AttributeTok{ID =} \DecValTok{1}\SpecialCharTok{:}\DecValTok{3}\NormalTok{, }\AttributeTok{code =}\NormalTok{ letters[}\DecValTok{1}\SpecialCharTok{:}\DecValTok{3}\NormalTok{])}

\CommentTok{\# Creates tables that follow OUP guidelines using xtable}
\FunctionTok{library}\NormalTok{(xtable)}
\FunctionTok{print}\NormalTok{(}\FunctionTok{xtable}\NormalTok{(df, }\AttributeTok{caption =} \StringTok{"This is the table caption"}\NormalTok{, }\AttributeTok{label =} \StringTok{"tab:tab1"}\NormalTok{),}
  \AttributeTok{comment =} \ConstantTok{FALSE}
\NormalTok{)}
\end{Highlighting}
\end{Shaded}

\begin{table}[ht]
\centering
\begin{tabular}{rrl}
  \hline
 & ID & code \\ 
  \hline
1 &   1 & a \\ 
  2 &   2 & b \\ 
  3 &   3 & c \\ 
   \hline
\end{tabular}
\caption{This is the table caption} 
\label{tab:tab1}
\end{table}

You can reference this table as follows: Table \ref{tab:tab1}.

\subsection{\texorpdfstring{Generate a table using
\texttt{kable}}{Generate a table using kable}}\label{generate-a-table-using-kable}

\begin{Shaded}
\begin{Highlighting}[]
\NormalTok{df }\OtherTok{\textless{}{-}} \FunctionTok{data.frame}\NormalTok{(}\AttributeTok{ID =} \DecValTok{1}\SpecialCharTok{:}\DecValTok{3}\NormalTok{, }\AttributeTok{code =}\NormalTok{ letters[}\DecValTok{1}\SpecialCharTok{:}\DecValTok{3}\NormalTok{])}

\CommentTok{\# kable can alse be used for creating tables}
\NormalTok{knitr}\SpecialCharTok{::}\FunctionTok{kable}\NormalTok{(df,}
  \AttributeTok{caption =} \StringTok{"This is the table caption"}\NormalTok{, }\AttributeTok{format =} \StringTok{"latex"}\NormalTok{,}
  \AttributeTok{booktabs =} \ConstantTok{TRUE}\NormalTok{, }\AttributeTok{label =} \StringTok{"tab2"}
\NormalTok{)}
\end{Highlighting}
\end{Shaded}

\begin{table}

\caption{\label{tab:tab2}This is the table caption}
\centering
\begin{tabular}[t]{rl}
\toprule
ID & code\\
\midrule
1 & a\\
2 & b\\
3 & c\\
\bottomrule
\end{tabular}
\end{table}

You can reference this table as follows: Table \ref{tab:tab2}.

\section{Discussion}\label{discussion}

You can cross-reference sections and subsections as follows: Section
\ref{materials-and-methods} and Section \ref{a-subsection}.

\textbf{\emph{Note:}} the last section in the document will be used as
the section title for the bibliography.

\section{References}\label{references}

\section{Appendix}\label{appendix}

\paragraph{PC Specificiations}\label{pc-specificiations}

\hfill\break

\begin{tabular}{l|r|r|r}
\hline
  & CPU & MEMORY & OPERATING SYSTEM\\
\hline
Ishango PC &  9th Gen Intel® Core™ i3-9100  & 8,0 GB &  Ubuntu 22.04\\
\hline
MidddleTROld & 9th Gen Intel(R) Core(TM) i5-9500 CPU & 8,0 GB &  Windows 10\\
\hline
MiddleTRNew & 12th Gen Intel(R) Core(TM) i5-13400 & 16,0 GB &  Windows 10\\
\hline
ScilabB &  12th Gen Intel(R) Core(TM) i5-12400 &  16,0 GB  &  Windows 10\\
\hline
Surface &  9th Gen Intel(R) Core(TM) i5-8250 CPU &  16,0 GB  &  Windows 10\\
\hline
ASUS laptop &  5th Gen Intel(R) Core(TM) i7-5500U CPU &  6,0 GB  &  Windows 10\\
\end{tabular}

\hfill\break


\begin{notes}[Acknowledgements]
This is an acknowledgement.

It consists of two paragraphs.
\end{notes}




\end{document}
